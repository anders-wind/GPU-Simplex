% !TeX spellcheck=en_GB
\section{Introduction}
Linear programming is one of the most used methods in optimisation theory. The Simplex algorithm (hereafter Simplex) is a widely used algorithm for solving linear programs. While Simplex has worst-case exponential running time, it is very fast in practice and is still used in most of the off-the-shelf linear programming frameworks.

For a lot of problems, using a single linear program might become impractical due to the size of the input, or problems might contain subproblems which are completely independent and therefore solvable in parallel. Multiple researchers and practitioners have made both CPU and GPU parallel versions of Simplex, but to our knowledge none of them work on parallelism across multiple Simplex instances.

\newpar In this project, we implement three different versions of massively parallel Simplex on multiple instances. The implementations are written in the data-parallel, purely functional language Futhark, which compiles to highly efficient GPU code. Furthermore, we benchmark and compare the different parallel algorithms to assess the possible speed-ups as well as compare them with CPU versions of Simplex. We focus on how flattening can be used to achieve higher levels of parallelism and how well it scales with respect to the overhead created by the flattening techniques.

