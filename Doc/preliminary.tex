% !TeX spellcheck=en_GB
\section{Preliminary}
\subsection{Linear Programming}
A \textit{linear program}(LP) consists of a set of $n$ \textit{variables} and $m$ \textit{linear constraints}. The LP specifies an objective function which maximizes or minimizes a linear combination of the variables and their coefficients, subject to the m constraints. All variables and constraints must be linear and therefore it is not allowed to multiply variables with each other or itself. To formulate precisely:

\begin{alignat}{3}
\text{max: } &\sum_{i=1}^{n} c_i x_i\nonumber\\ 
\text{s.t }  & \sum_{j=1}^{n} a_{ij} x_j \leq b_i && \text{ for } i=1,2,...,m\nonumber\\ 
& x_j \geq 0                         && \text{ for } j=1,2,...,n\nonumber
\end{alignat}

The variables can be seen as dimensions in coordinate system and the constraints defines a convex feasibility region where the values of the variables satisfies all of the constraints. An extreme point of the region will have a specific objective value and it can be shown that the optimal value of the linear program can be found in such a point. 

The result of running a linear program can either be a vector of the variables or a scalar representing the resulting value of the objective function depending on what is wanted. 

By adding the constraint that the variables have to be integers the linear programming problem becomes NP-hard. This version is called Integer Linear Programming and can be used to solve other NP-hard problems quite efficiently.

\subsection{Simplex}
Simplex is a specific algorithm for finging the optimal value

\subsection{Flattening}
